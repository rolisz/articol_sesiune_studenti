Table \ref{table:seg_values} contains the average, maximum and minimum values obtained for the F1 measure\cite{fawcett2006introduction} of the character segmentation problem. The F1 measure is used instead of the accuracy because the two classes are imbalanced: there are 11475 data points which indicate a segmentation point, while there are 19402 points which are not segmentation points, almost twice as many. 

\begin{table}[h]
\caption{The F1 score for the character segmentation experiment}
\label{table:seg_values}
\begin{tabular}{llllll}
\hline
Number of trees & Number of features & Min     & Max     & Mean    & Std. dev. \\ \hline
150 & 20 & 0.87318 & 0.88326 & 0.87735 & 0.00363 \\ 
200 & 20 & 0.86975 & 0.88312 & 0.87657 & 0.00490 \\ 
250 & 20 & 0.87258 & 0.88776 & 0.87773 & 0.00531 \\ 
150 & 8 & 0.86894 & 0.88376 & 0.87569 & 0.00517 \\ 
200 & 8 & 0.87227 & 0.88299 & 0.87675 & 0.00413 \\ 
250 & 8 & 0.87178 & 0.88470 & 0.87717 & 0.00426 \\ 
150 & 120 & 0.87262 & 0.88631 & 0.87816 & 0.00476 \\ 
200 & 120 & 0.87122 & 0.88387 & 0.87724 & 0.00418 \\ 
250 & 120 & 0.87367 & 0.88565 & 0.87885 & 0.00391 \\ 
150 & 40 & 0.87073 & 0.88671 & 0.87822 & 0.00552 \\ 
200 & 40 & 0.86980 & 0.88519 & 0.87845 & 0.00513 \\ 
250 & 40 & 0.87358 & 0.88671 & 0.87936 & 0.00445 \\  \hline
\end{tabular}
\end{table}

Table \ref{table:seg_conf} contains the confusion matrix for the best experiment on the character segmentation problem.

\begin{table}[h]
\caption{The confusion matrix for the character segmentation experiment}
\label{table:seg_conf}
\begin{tabular}{lll}
\hline
 & No split & Split \\ \hline
Predicted no split & 4556 & 363 \\ 
Predicted split & 255 & 2546 \\  \hline
\end{tabular}
\end{table}