\section{Literature review}
Optical character recognition has long been of interest in the computer industry. In 1976, Ray Kurzweil presented an omni-font OCR engine, that together with a speech synthesiser would read texts to blind people. It came together with its own special scanner, that was required for taking the images of pages. \cite{schantz1982history}

R. Holley present in a study\cite{Holley_2009} done to evaluate OCR on australian newspapers that the accuracy of various commercial software on newspapers varies from 71\% to 98\%. A similar study done by E. Klijn on dutch newspapers reported accuracies varying between 68\% to 99\%, with this number depending mostly on the quality of the image scan and on the state of the newspaper. 

One of the most popular machine learning datasets is the MNIST dataset of handwritten digits\cite{lecun1998mnist}. This dataset contains 60000 examples to be used for training and 10000 to be used as a test set. All the black and white (bilevel) images in this data set were initially size normalized to fit a 20x20 bounding box, while preserving aspect ratio. The resulting images were anti-aliased, which introduced grey levels. In the end, the images were centered in 28x28 image, translating the center of mass of the pixels to the center of the 28x28 image. 

Initial results on the MNIST database, from the paper published by Y. LeCun and L. Bottou in 1998\cite{Lecun_1998}, are summarized in Table \ref{table:mnist_results}. 

\begin{table}[h]
\caption{Test error rates obtained on MNIST using various algorithms}
\label{table:mnist_results}
\begin{tabular}{ll}
\hline
Classifier                   & Test error rate \\ \hline
1-layer neural network       & 12.0            \\
K-nearest-neighbors          & 5.0             \\
PCA + quadratic classifier   & 3.3             \\
SVM with Gaussian Kernel     & 1.4             \\
2-layer neural network       & 4.7             \\
3-layer neural network       & 2.45            \\
Convolutional neural network & 0.95            \\ \hline
\end{tabular}
\end{table}
OCR 

To our knowledge, there are no existing approaches for receipts.