\section{Introduction}
Optical character recognition (OCR) is the ability of the computer to extract textual information from image data, such as pixels. This is useful because by introducing the data available on paper, we can use a computer to process, index and search the data much faster.  

OCR is a difficult problem, even when done on straight papers, without creases, that are scanned, because first we must identify letters on the page and distinguish them from tables, figures and other objects that might be there, and because there are many kinds of fonts that have to be recognized. The problem of OCR on photographed documents is even more difficult. The illumination can vary, the document might be curved, there might be a skewed perspective and so on. 



Information extraction

ce abordez, complexitate
The paper aims at 
The rest of the paper is structured as follows.