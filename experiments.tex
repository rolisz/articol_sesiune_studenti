\section{Experimental evaluation}
\subsection{Data set and processing}
The data set that was obtained from 20 receipts that were manually annotated with the position of each character in them. In total there are 7045 characters. There are 74 different characters, including digits, uppercase and lowercase letters and punctuation. 

The bounding boxes of the characters were extracted from the images. The resulting patches were normalized to have a size of 30x30 pixels and were converted to grayscale. The small images that resulted after this processing were used as the data set for the character recognition problem.

For the character segmentation problem, positive and negative patches were extracted from the images, each containing 40 columns of pixels. The positive example were obtained by taking the leftmost and rightmost columns of the bounding boxes of characters, together with 19 previous columns and 20 columns that followed. The negative examples were obtained by sampling randomly from the middle of a character and taking 19 columns from before and 20 from after.

\subsection{Training and testing}
The data set was shuffled and then split into two parts, one for training and one for testing. The splitting was done in a random way, because the data points are independent and order does not matter. The training set contained 80\% of the data and the test set contained the remaining 20\%. 

For the character recognition problem, the labels corresponding to each character were converted to a 

\subsection{Results}

