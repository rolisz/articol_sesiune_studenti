\section{Methodology}
This section presents the background of the machine learning approaches used in the problem of OCR and then the specifics of the models applied to this problem are discussed. 
\subsection{Theoretical background}
\subsubsection{Random forests}
Random forests have been introduced by Leo Breiman and Adele Cutler\cite{breiman2001random} as an ensemble of decision trees. When using only one decision tree to make a classification, one often runs into problems with high variance or high bias. Random forests present a mechanism to avoid these problems to make more accurate models, that generalize better. 

When training the random forest, for each tree, n samples are taken with replacement from the training data (a bootstrap sample) and the tree is trained on these, using a slightly modified procedure. When splitting in a node, instead of choosing the best split across all features, as is done in the classical decision tree approach, here the best split is chosen from a random subset of the m features. This is done to avoid correlations between trees: otherwise good features would always get picked in all trees, so they would be correlated. As m is smaller, the correlation between the trees is smaller, but the strength of each tree (how well it can predict on its own) also goes down, so a balance must be found these extremes. 

Random forests are a popular algorithm in many machine learning competitions, because they are fast, they don't have many parameters to tune, yet still produce good predictions. Among their weaknesess is the fact they can easily overfit a noisy dataset. 
\subsubsection{Linear SVM}

\subsection{Model design}



