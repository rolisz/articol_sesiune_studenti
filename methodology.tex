\section{Methodology}
\label{sec:method}
This section presents the background of the machine learning approaches used in the problem of OCR and then the specifics of the models applied to this problem are discussed. 

\subsection{Theoretical background}

\subsubsection{Random forests}

Random forests have been introduced by Leo Breiman and Adele Cutler\cite{breiman2001random} as an ensemble of decision trees. When using only one decision tree to make a classification, one often runs into problems with high variance or high bias. Random forests present a mechanism to avoid these problems to make more accurate models, that generalize better. 

When training the random forest, for each tree, n samples are taken with replacement from the training data (a bootstrap sample) and the tree is trained on these, using a slightly modified procedure. When splitting in a node, instead of choosing the best split across all features, as is done in the classical decision tree approach, here the best split is chosen from a random subset of the m features. This is done to avoid correlations between trees: otherwise good features would always get picked in all trees, so they would be correlated. As m is smaller, the correlation between the trees is smaller, but the strength of each tree (how well it can predict on its own) also goes down, so a balance must be found these extremes. 

Random forests are a popular algorithm in many machine learning competitions, because they are fast, they don't have many parameters to tune, yet still produce good predictions. Among their weaknesess is the fact that they can easily overfit a noisy dataset. 

\subsubsection{Support vector machines}

Support vector machines\cite{Cortes_1995} are discriminative classifiers formally defined by high-dimensional hyperplanes, which are used to distinguish between the classes to which data points belong. The hyperplane defined by an SVM maximizes the margin to the data points used in training, hoping that this leads to a better generalization of the classifier. 

SVM can be used to project the original data to a much higher dimensional space (in some cases even infinte-dimensional space), so that they are linearly separable in that space. The mapping to a higher dimensional space is done using a kernel function. 

One of the more popular kernels\cite{Chang:2010:TTL:1756006.1859899} that can be used is the radial basis function (RBF) kernel, which is defined as:

\[

    K(\mathbf{x}, \mathbf{x'}) = \exp\left(-\frac{||\mathbf{x} - \mathbf{x'}||_2^2}{2\sigma^2}\right) 
\]

The value of the RBF value goes from zero (at infinity) to one (when $ x = x'$), so it can be viewed as a similarity measure between the two samples. \cite{Vert}

Because sometimes there is noise in the data, so it may not be possible to separate the data linearly, not even in a high dimensional space or, even if possible, this may not be desirable, because it would overfit to the data and not generalize well. In such cases, it is prefered to have a decision surfaces that makes some mistakes on the training data, but generalizes better and represents the noisy data more accurately. SVMs can be used as soft margin classifiers, allowing examples to be classified wrongly at training time, but penalizing them according to their distance to the other side of the hyperplane. \cite{russell1995artificial}

Because SVMs separate only two classes, when there are multiple classes to be distinguished, the "one-vs-all" approach can be used for classficiation, and is as accurate as any other approach for this problem\cite{rifkin2004defense}. In this case, one classifier is trained for each class, to distinguish it from all other classes. To make a prediction, all classifiers predict their value and the one that is used will be the one with the highest confidence score.

\subsection{Model design}

\subsubsection{The random forest model}
The random forest was used as a model for the character segmentation problem. The criterion for choosing the best feature to split a node is the information gain (entropy). Trees are grown to their full depth, no pruning or limitation is applied to the branches. 

The other parameters of the random forest were chosen by cross-validation: the number of trees and the number of features to consider when randomly sampling from the feature space. 

\subsubsection{The support vector machine model}
The SVM was used for the character recognition problem. The performance of both linear and radial basis function kernels was evaluated. 

The regularization parameter of the SVM was determined using cross-validation. 

\subsection{Document Layout Analysis}
Before any characters can be recognized in a receipt, the image must first be preprocessed and normalized. This is done in several steps. 

The preprocessing consists of binarizing the images, using Otsu's method\cite{otsu1975threshold}, which adapts the threshold based on the histogram of the image. This step is done to remove any noise and speckles from the image. 

The first step in normalization is to straighten the images. The receipts are assumed to be photographed with a mobile phone camera. Users will most often take photos that are slightly rotated. The orientation of the images is assumed to be vertical, so the software will not try to identify if the receipt is horizontal. To straighten the images, they are rotated from -10 to 10 degrees, with a 0.3 angle step, and a horizontal projection (summing the pixel values row-wise) is done for each resulting image. The straight image is assumed to be the one were are the most variations between the peaks and valleys of the histogram, because in the straight image there would be high peaks because of the lines and low valleys because of the space between lines. 

The following step is removing the edges of the image, to keep only the receipt, removing any background. Due to variations in illumination, we cannot simply look for white patches to identify the receipt, because mobile cameras often use flash which gives receipts a blue tint while photos taken indoor close to a source of light have a yellow hue. The approach that was used was to look at the horizontal and vertical projections and to remove the section from the top and bottom that is over a threshold. 

The last step is detecting the lines in the receipt. Because the images are already straight and without edges, all we have to do is identify the peaks in the horizontal histogram in the image.  